% vim: filetype=tex spell
%CubeSats are becoming all the rage with universities these days. There small form factor and low mass put restrictions on the mass and power budgets that make the conventional attitude control mechanisms less attractive. Many CubeSats with attitude control use ether active or passive magnetic systems. This thesis describes the implementation of a system that provides better control than passive systems with lower power consumption then active systems.

%Magnetic attitude control for CubeSats comes in one of two forms active or passive. Passive systems use permanent magnets and hysteresis material which draw no power from the CubeSat but the orientation is relative to the local magnetic field. Active systems generally use coils that produce torque only when current is driven through them allowing better control of the attitude of the CubeSat while using more power. The \ac*{ARC} will have an attitude control system that uses coils with a hard magnetic core which allows torque to be generated without a constant expenditure of energy but allows the torque to be changed. To achieve proper attitude alignment a biasing scheme is used which removes the need for absolute attitude knowledge.

%The \acf*{ARC} is a CubeSat which is being built at the \acf*{UAF}. The spacecraft utilizes a unique atitude control system which uses hard magnetic torquers for atitude control. 


%Magnetic attitude control systems fall into two main categories: active and passive. Active control is often achieved by running current through a coil to produce a dipole moment, while passive control uses the dipole moment from permanent magnets that consume no power. This presentation describes a system that uses twelve hard magnetic torquers along with a magnetometer to achieve three-axis attitude alignment. The torquers only consume current when their dipole moment is flipped, thereby significantly reducing power requirements compared with traditional active control.  Final attitude alignment is achieved using a dual axis magnetic dipole moment bias algorithm. The system has been shown, in simulation, to be capable of detumbling and aligning a 1U CubeSat to within 5° of a nadir facing attitude in a high inclination orbit. The presentation will focus on design, testing and fabrication of CubeSat hardware in preparation for a proposed launch in 2013.

In recent years there has been a growing interest in smaller sattelites. Smaller sattelites are cheaper to build and launch then larger sattelites. One form factor, the CubeSat, is especially popular with universites and is a 10cm cube. Being smaller means that the mass and power budgets are tighter and as such new ways must be developed to cope with these constraints. Traditional attitude control systems often use reaction wheels with gass thrusters which are not possible on a CubeSat. Most CubeSats use magnetic attitude controll which uses the Earth's magnetic field to torque the sattelite into the propper orientation. Magnetic attitude control systems fall into two main categories: active and passive. Active control is often achieved by running current through a coil to produce a dipole moment, while passive control uses the dipole moment from permanent magnets that consume no power. This thesis describes a system that uses twelve hard magnetic torquers along with a magnetometer to achieve three-axis attitude alignment. The torquers only consume current when their dipole moment is flipped, thereby significantly reducing power requirements compared with traditional active control.  Final attitude alignment is achieved using a dual axis magnetic dipole moment bias algorithm. The system has been shown, in simulation, to be capable of detumbling and aligning a 1U CubeSat in a nadir facing attitude in a high inclination orbit. The thesis will focus on design, testing and fabrication of CubeSat hardware in preparation for a proposed launch in 2013.

