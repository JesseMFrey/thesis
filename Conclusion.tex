% vim: filetype=tex spell

\chapter{Conclusion and Future work}

\section{Conclusion}

The goal of this thesis was to make a working prototype of the hardware described in \cite{Mentch11}. Flight hardware has been built with software to detumble the satellite. The alignment phase described in \cite{Mentch11} was not implemented primarily due to the reasons stated in \cref{ch:BG} but in addition the orbit in \cite{Mentch11} was circular and for the \ac{ARC}1 mission the orbit will be elliptical.

In a circular orbit the angular velocity required to remain nadir pointing is constant throughout the orbit. In an elliptical orbit the angular velocity to remain nadir pointing is not constant throughout the orbit. The changing angular velocity presents a problem because the algorithm in \cite{Mentch11} works by setting the angular velocity of the satellite to be constant. 

A large part of the work for this thesis was focused on the torquer compesation. It was sugested that torquer compensation was possible but it was not mentioned in \cite{Mentch11}. Software has been written to interface with the \ac{LEDL} board to read magnetometer data and compensate for the torquer field. The torquer compensation has been tested and found to accurately measure the external magnetic field well enough to compute the attitude to within \textpm{}5\textdegree.

\section{Future Work}

The system built for this thesis does not fully implement everything described in \cite{Mentch11}. The main shortcomings of the described system is the software. The hardware is essentially the same as described in \cite{Mentch11} except for the torquer sizing.

The torquers as described in \cite{Mentch11} are not feasible. The torquer flips in \cref{fig:satWV} showed about a \textpm{}0.5\% variation across multiple flips. The vernier torquers described in \cite{Mentch11} were $\sfrac{1}{200}^{\mathrm{th}}$, or 0.5\%, the strength of the primary torquers. The result is that a set of primary torquers that is set to produce zero torque could produce the same amount of torque as a pair of vernier torquers. The solution used in this thesis was to not use the vernier torquers and fill the empty slots with primary torquers. This could be improved upon if a torquer core could be made with approximately 10\% the dipole moment of the alnico torquers. 

The rotation rate determination algorithm was found to be a much more complex problem than described in \cite{Mentch11} which suggested that rotation rates could be determined directly from magnetic field measurements. Upon further investigation it was found that this was not feasible for the alignment phase. As an alternative the rotation rates can be determined from the magnetic field using a Kalman filter. The Kalman filter brings an additional level of complexity that the original system attempted to avoid. 

The Kalman filter uses a magnetic field model for attitude determination. The magnetic field model uses the location of the satellite to calculate the expected local magnetic field in the orbital reference frame. This requires that the position of the satellite be known for the Kalman filter to work. The location will be calculated by uplinking the orbital parameters of the satellite and using an orbit propagator to calculate position as the satellite moves through its orbit.

The bias window were originally supposed to be determined using the magnetic field to estimate the latitude. With the Kalman filter the location can be directly determined using the orbit propagator so there is no need for latitude determination from the Earth's magnetic field.

\subsection{New Concept of Operations}

The algorithm shown in \cref{fig:swblock} with additions is shown in \cref{fig:swblock-future}. The orbital parameters are used to determine the bias window and to determine the local magnetic field direction. The Kalman filter is used to output rates to the torque algorithm. The bias window is used by the torque algorithm to determine when and how to bias the torquers.

\begin{figure}[H]
    \centering
    \begin{tikzpicture}[node distance = 0.5cm, auto]
    % Place nodes
    \node [block] (field) {Magnetic Field Measurements};
    \node [block,right=of field] (cal) {Torquer offset correction};
    \node [block,node distance=0.7cm,right=of cal] (alg) {Torque algorithm};
    \node [block, right=of alg] (q) {Torque Quantization};
    \node [block,above=of cal] (igrf) {Magnetic field model};
    \node [block,node distance=0.7cm,right=of igrf] (rates) {Kalman Filter};
    \node [block,left=of igrf] (pos) {Orbit Propagator};
    \node [block, above=of q] (win) {Bias Window Determination};
    \node [block, right=of q] (choose) {Choose Torquers to fire};
    \node [block,node distance=0.7cm, right=of choose] (fire) {Fire Torquers};
    \node [block, below=of choose] (mem) {Torquer Status Tracking};
    \node [block,node distance=0.86cm, below=of fire] (sens) {Torquer feedback};

    %draw lines
    \path [conn] (field) -- (cal);
    \path [conn] (cal.east) -- ++(0.3cm,0) |- (rates.190);
    \path [conn] (cal) -- (alg);
    \path [conn] (rates) -- (alg);
    \path [conn] (win) -- (alg);
    \path [conn] (alg) -- (q);
    \path [conn] (q) -- (choose);
    \path [conn] (choose) -- (fire);

    \path [conn] (choose.east) -- ++(0.3cm,0) |- (mem.10);
    \path [phconn] (fire) -- (sens);

    \path [conn] (sens.west) -- ++(-0.3cm,0) |- (mem);
    \path [conn] (mem) -- (choose);
    \path [conn] (mem) -| (cal);

    \path [conn] (pos) -- (igrf);
    \path [conn] (igrf.east) -- ++(0.3cm,0) |- (rates.west);

    \path [conn] (pos) |- ([yshift=0.3cm]win.north) -- (win);

    \end{tikzpicture}
    \caption{Overall Software Block Diagram}
    \label{fig:swblock-future}
\end{figure}

The torque algorithm varies depending on the current mode. In mode 1 the Kalman filter and bias window determination are not used and the algorithm is the one showin in \cref{fig:swblock}. In modes 2 and 3 the algorithms shown in \cref{fig:mode2,fig:mode3} are used.

\subsubsection{Mode 2}

\Cref{fig:mode2} shows the Mode 2 torque algorithm block diagram. Torque is calculated the same way as in Mode 1 but this time a bias is added depending on which region of the orbit the satellite is in. The bias tends to cause the satellite to rotate. This causes the algorithm to cancel out the bias. This is prevented by preventing the resulting dipole moment from being in the opposite direction as the bias. Outside of the bias regions the torquers are set to produce no torque.

\begin{figure}[H]
    \centering
    \begin{tikzpicture}[node distance = 0.5cm, auto]
    % Place nodes
    \node [input] (field) {Magnetic Field};
    \node [block, right=of field] (alg) { $k \frac{\vect{\omega}_{err} \cross \vect{B}}{\vect{B} \cdot \vect{B}}$ };
    \node [input, right=of alg] (rates) {Rotation Rates};
    \node [oppr,node distance=1cm,below=of alg] (sum) {+};
    \node [block,node distance=1.2cm,left=of sum] (bias) {Mode 2 bias table};
    \node [input,left=of bias] (win) {Bias Window};
    \node [block,node distance=1cm,below=of sum] (fix) {Bias Fix};
    \node [block,below=of fix] (coast) {Coast?};
    \node [point, below=of coast] (out) {};

    %draw lines
    \path [conn] (field) -- (alg);
    \path [conn] (rates) -- (alg);
    \path [conn] (alg) -- (sum);
    \path [conn] (win) -- (bias);
    \path [conn] (bias) -- (sum);
    \path [conn] (sum) -- (fix);
    \path [conn] (fix) -- (coast);
    \path [conn] (win) |- (coast);
    \path [conn] (coast) -- (out);
    \path [conn] (bias) |- (fix);

    \end{tikzpicture}
    \caption{Mode 2 Torque Algorithm Block Diagram}
    \label{fig:mode2}
\end{figure}

\subsubsection{Mode 3}

\Cref{fig:mode3} shows the Mode 3 torque algorithm block diagram. This is similar to Mode 2 except only the north pole bias window is used and outside the bias window torque is generated to slow rotation rates.

\begin{figure}[H]
    \centering
    \begin{tikzpicture}[node distance = 0.5cm, auto]
    % Place nodes
    \node [input] (field) {Magnetic Field};
    \node [block, right=of field] (alg) { $k {\frac{\vect{\omega}_{err} \cross \vect{B}}{\vect{B} \cdot \vect{B}}}$ };
    \node [input, right=of alg] (rates) {Rotation Rates};
    \node [oppr,node distance=1cm,below=of alg] (sum) {+};
    \node [block,node distance=1.2cm,left=of sum] (bias) {Mode 3 bias table};
    \node [input,left=of bias] (win) {Bias Window};
    \node [block,node distance=1cm,below=of sum] (fix) {Bias Fix};
    \node [point, below=of fix] (out) {};

    %draw lines
    \path [conn] (field) -- (alg);
    \path [conn] (rates) -- (alg);
    \path [conn] (alg) -- (sum);
    \path [conn] (win) -- (bias);
    \path [conn] (bias) -- (sum);
    \path [conn] (sum) -- (fix);
    \path [conn] (fix) -- (out);
    \path [conn] (bias) |- (fix);


    \end{tikzpicture}
    \caption{Mode 3 Torque Algorithm Block Diagram}
    \label{fig:mode3}
\end{figure}

\subsection{Alternate operations}

The original system presented in \cite{Mentch11} used the biasing scheme to get the satellite into the correct attitude without ever knowing the attitude. The Kalman filter proposed for rate determination must compute attitude in order to compute the rotation rates. Because the attitude is known a more traditional attitude control algorithm to be used. Using a more traditional algorithm will allow the \ac{ACDS} to function in orbits that are elliptical. 


