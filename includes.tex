% vim: filetype=tex spell

\usepackage{amsmath, amssymb, amsfonts} % Thanks, AMS!
\usepackage{xfrac}
\usepackage{graphicx} % Graphics stuff
\usepackage{textcomp}
\usepackage{array}
\usepackage{fixltx2e} % Allows \(\) in captions, amongst other things.
\usepackage{pxfonts} % The Paladino font
\usepackage{verbatim}
\usepackage{amsmath}
\usepackage{amssymb,amsfonts,textcomp}
\usepackage{array}
\usepackage[printonlyused,withpage]{acronym}
\usepackage{tocloft}
\usepackage{topsection}
\usepackage{pdflscape}
\usepackage{enumitem}
\usepackage{pdflscape}
\usepackage{afterpage}
\usepackage{xspace}
\usepackage{csquotes}
\usepackage[super]{nth}
\usepackage[usenames,dvipsnames]{xcolor}
\usepackage{float}
%\synctex=1
\usepackage[square,numbers,comma,sort&compress]{natbib}

\def\xcolorversion{2.00}

\usepackage[version=latest]{pgf}
\usepackage{xkeyval,calc,tikz,fp}
\usepackage{listings}
\usepackage[textwidth=5em,disable]{todonotes}

%include hyperref last so that it can modify some commands
\usepackage{hyperref}
%add hypcap after hyperref
\usepackage[all]{hypcap}
%include cleverref after hyperref because it says to
\usepackage[noabbrev]{cleveref}

%blue links good for on screen
\hypersetup{linktoc=all,colorlinks,linkcolor=blue,pdfauthor={Jesse Frey},pdfstartview=FitH}
%Black links good for printing
%\hypersetup{bookmarks,colorlinks,citecolor=black,urlcolor=black,filecolor=black,linkcolor=black,pdfauthor={Jesse Frey},pdfusetitle,pdfstartview=FitH}

%add figures folder to graphics path
\graphicspath{{./Figures/}}

%libraries for tikz
\usetikzlibrary{chains}
\usetikzlibrary{decorations.shapes}
\usetikzlibrary{decorations.markings}
\usetikzlibrary{shapes}
\usetikzlibrary{positioning}
\usetikzlibrary{backgrounds}
\usetikzlibrary{arrows.meta}
\usetikzlibrary{fit}

%define colors for matlab listings
\definecolor{comentcolor}{RGB}{28,172,0} % color values Red, Green, Blue
\definecolor{stringcolor}{RGB}{170,55,241}

%indent all listings
\lstset{
    xleftmargin=.5in,
    xrightmargin=.5in} 

\newcommand\digitstyle{\color{BrickRed}}
\makeatletter
\newcommand{\ProcessDigit}[1]
{%
  \ifnum\lst@mode=\lst@Pmode\relax%
   {\digitstyle #1}%
  \else
    #1%
  \fi
}

%setup listings for Matlab for listings
\lstdefinestyle{code}{
    basicstyle=\ttfamily,
    breaklines=true,%
    morekeywords={matlab2tikz},
    keywordstyle=\color{blue},%
    morekeywords=[2]{1}, keywordstyle=[2]{\color{black}},
    identifierstyle=\color{black},%
    stringstyle=\color{stringcolor},
    commentstyle=\color{comentcolor},%
    showstringspaces=false,%without this there will be a symbol in the places where there is a space
    numbers=left,%
    numberstyle={\color{BurntOrange}},% size of the numbers
    numbersep=9pt, % this defines how far the numbers are from the text
    emph=[1]{for,end,break},emphstyle=[1]\color{red}, %some words to emphasise
    %emph=[2]{word1,word2}, emphstyle=[2]{style},    
    literate=
        {0}{{{\ProcessDigit{0}}}}1
        {1}{{{\ProcessDigit{1}}}}1
        {2}{{{\ProcessDigit{2}}}}1
        {3}{{{\ProcessDigit{3}}}}1
        {4}{{{\ProcessDigit{4}}}}1
        {5}{{{\ProcessDigit{5}}}}1
        {6}{{{\ProcessDigit{6}}}}1
        {7}{{{\ProcessDigit{7}}}}1
        {8}{{{\ProcessDigit{8}}}}1
        {9}{{{\ProcessDigit{9}}}}1
        {.0}{{{\ProcessDigit{.0}}}}2
        {.1}{{{\ProcessDigit{.1}}}}2
        {.2}{{{\ProcessDigit{.2}}}}2
        {.3}{{{\ProcessDigit{.3}}}}2
        {.4}{{{\ProcessDigit{.4}}}}2
        {.5}{{{\ProcessDigit{.5}}}}2
        {.6}{{{\ProcessDigit{.6}}}}2
        {.7}{{{\ProcessDigit{.7}}}}2
        {.8}{{{\ProcessDigit{.8}}}}2
        {.9}{{{\ProcessDigit{.9}}}}2
        {-0}{{{\ProcessDigit{-0}}}}2
        {-1}{{{\ProcessDigit{-1}}}}2
        {-2}{{{\ProcessDigit{-2}}}}2
        {-3}{{{\ProcessDigit{-3}}}}2
        {-4}{{{\ProcessDigit{-4}}}}2
        {-5}{{{\ProcessDigit{-5}}}}2
        {-6}{{{\ProcessDigit{-6}}}}2
        {-7}{{{\ProcessDigit{-7}}}}2
        {-8}{{{\ProcessDigit{-8}}}}2
        {-9}{{{\ProcessDigit{-9}}}}2
}

%setup listings for Matlab for listings
\lstdefinelanguage{verbatim}{
    sensitive=false,
}

\lstdefinestyle{verbatim}{
    numbers=none,
    basicstyle=\ttfamily,
    columns=fullflexible,
    keepspaces=true,
}

