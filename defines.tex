% vim: filetype=tex spell

%define some math commands
\newcommand{\vect}[1]{\overrightarrow{#1}}
\newcommand{\cross}[0]{\times}
\newcommand{\matt}[1]{\boldsymbol{#1}}
\newcommand{\transpose}{^T}
\newcommand{\unit}[1]{\, \mathrm{#1}}

%highlight future work
\newcommand{\future}[1]{{\color{red}\itshape#1}}

\pgfdeclareshape{inshape}
{%
  % All anchors are taken from the 'rectangle' shape:
  \inheritsavedanchors[from={rectangle}]%
  \inheritanchor[from={rectangle}]{center}%
  \inheritanchor[from={rectangle}]{north}%
  \inheritanchor[from={rectangle}]{south}%
  \inheritanchor[from={rectangle}]{west}%
  \inheritanchor[from={rectangle}]{east}%
  \inheritanchorborder[from={rectangle}]%
  %
  % Only the background path is different
  %
  \backgroundpath{%
    % First the existing 'circle' code:
    \pgfmathsetlength{\pgf@xb}{\pgfkeysvalueof{/pgf/outer xsep}}%
    \pgfmathsetlength{\pgf@yb}{\pgfkeysvalueof{/pgf/outer ysep}}%
    \ifdim\pgf@xb<\pgf@yb%
      \advance\pgfutil@tempdima by-\pgf@yb%
    \else%
      \advance\pgfutil@tempdima  by-\pgf@xb%
    \fi%
    \pgfpathcircle{\centerpoint}{\pgfutil@tempdima}%
    %
    % Now the | and -- lines:
    \pgfmoveto{\pgfpointadd{\centerpoint}{\pgfpoint{0pt}{\pgfutil@tempdima}}}%
    \pgflineto{\pgfpointadd{\centerpoint}{\pgfpoint{0pt}{-\pgfutil@tempdima}}}%
    \pgfmoveto{\pgfpointadd{\centerpoint}{\pgfpoint{\pgfutil@tempdima}{0pt}}}%
    \pgflineto{\pgfpointadd{\centerpoint}{\pgfpoint{-\pgfutil@tempdima}{0pt}}}%
  }%
}

% Define block styles
\tikzstyle{decision} = [diamond, draw, fill=blue!20, text width=4.5em, text badly centered, inner sep=0pt]
\tikzstyle{block} = [rectangle, draw, fill=blue!20, text width=5em, text centered, rounded corners, minimum height=4em]
\tikzstyle{bigblock} = [rectangle, draw, fill=blue!20,rounded corners, minimum height=4em]
\tikzstyle{input} = [rectangle, draw, fill=green!20, text width=5em, text centered, rounded corners, minimum height=4em]
%\tikzstyle{input} = [inshape, draw, fill=green!20, text width=5em, text centered, minimum height=4em]
\tikzstyle{oppr} = [draw, circle,fill=blue!20,minimum height=2em]
\tikzstyle{cloud} = [draw, ellipse,fill=red!20, node distance=3cm,minimum height=2em]
%connections styles for flow charts
\tikzstyle{line} = [draw]
\tikzstyle{conn} = [draw,->]
\tikzstyle{phconn} = [draw,->,dashed]
%connection styles for block diagrams
\tikzstyle{bidr} = [draw,<->]
%\tikzstyle{flow} = [draw,--,decorate,decoration=triangles]
\tikzstyle{flow} = [draw,%
          decoration={%
            markings,%
            mark=at position 0.65 with {\arrow[scale=2]{stealth}},
          },postaction=decorate]

%point used as a dummy node for complicated arrows
\tikzstyle{point} = [circle,inner sep=0pt,minimum size=0pt,fill=none,node distance = 1.5cm]
%\tikzstyle{point} = [circle,inner sep=0pt,minimum size=2pt,fill=red,node distance = 1.5cm]     %alternate point for debugging

%style for annotating fiugres
\tikzstyle{na} = [baseline=-.5ex,remember picture]

%name of MATLAB
\newcommand{\matlab}{MATLAB\xspace}

