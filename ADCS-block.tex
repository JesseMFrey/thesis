% vim: filetype=tex spell 

% We need layers to draw the block diagram
\pgfdeclarelayer{background}
\pgfdeclarelayer{foreground}
\pgfdeclarelayer{Hardware}
\pgfdeclarelayer{CPU}

\pgfsetlayers{background,Hardware,CPU,main,foreground}

\begin{tikzpicture}
    %first draw ARCbus
    \begin{pgfonlayer}{Hardware}
        \node[draw=black,minimum height=1cm,minimum width=10cm,rotate=90] (arcbus) at (0,0) {\acs{ARC} Bus};
    \end{pgfonlayer}

    %\node [power] (capchg) {Capacitor\\charging\\circuit};
    %TODO: rotate text
    \node[perif,right of=arcbus]    (SPI_IIC)  {\acs{SPI} + \acs{I2C}};
    \node[prog,right of=SPI_IIC]    (command)  {command and control};
    \node[prog,right of=command]    (lpmt)     {\acs{LPMT} algorithm};
    \node[prog,below of=lpmt]       (housek)   {Housekeeping};
    \node[perif,below of=housek]    (SPI)      {\acs{SPI}};
    \node[hardware,below of=SPI]    (SD)       {\acs{SD}};
    %TODO: rotate text
    \node[perif,right of=lpmt]       (tqio)    {\acs{GPIO}};
    \node[hardware,right of=tqio]    (tqio)    {H-Bridges};
    \node [power,above of=lpmt]      (capchg) {Capacitors\\(one per axis)};
    \node [power,right of=capchg]    (cap) {Capacitors\\(one per axis)};
    %\path [Bus,powerL] (-3,-1.5) edge (arcbus);

    \node[hardware,left of=arcbus,node distance=4cm]  (LEDL)     {\acs{LEDL} \textmu{}C};
    \node[hardware,above of=LEDL]   (gyro)     {Gyros};
    \node[hardware,below of=LEDL]   (mag)      {Magnetomitor};

    %draw CPU boxes
    \begin{pgfonlayer}{CPU}
        \path[CPU] (SPI_IIC.west) +(0, 2cm) rectangle (tqio.east);
    \end{pgfonlayer}

    %draw
    \begin{pgfonlayer}{Hardware}
        %\path[draw=black] (3,3) rectangle (-3,-3);
        %\path[draw=black] (-4,-3) rectangle (-5,3);
        %\path[draw=black] (-6,3) rectangle (-8,0);
    \end{pgfonlayer}
\end{tikzpicture}


