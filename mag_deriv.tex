% vim: filetype=tex spell

\chapter{Calibration Equation Derivation}

\label{ch:mag-deriv}

\Cref{eq:magcross2} shows the equation from \cite{AN215} as shown in \cref{eq:magcross}.

\begin{equation}
    V_s = V_b \left(S_s H_s + D \cdot H_c + V_{os} \right)
    \label{eq:magcross2}
\end{equation}

The substitution shown in \cref{eq:adcsub} is made to get \cref{eq:magcross-sub}

\begin{align}
    V_s' & =\frac{V_s}{V_b}\\
    V_s' & = S_s H_s + D \cdot H_c + V_{os}
    \label{eq:magcross-sub}
\end{align}

To solve for $H_c$ and $H_s$ \cref{eq:magcross-sub} is duplicated for the cross axis direction to get \cref{eq:add-cross}.

\begin{equation}
    \begin{aligned}
        V_s' - {V_{os}}_s &=  {S_s}_s H_s  + D_s \cdot H_c\\
        V' - {V_{os}}_s &=  {S_s}_c H_c  + D_c \cdot H_s
    \end{aligned}
    \label{eq:add-cross}
\end{equation}

To solve the equations are put into matrix form as shown in \cref{eq:eqmat}.

\begin{equation}
    \begin{bmatrix}
        V'_s - {V_{os}}_s\\
        V'_c - {V_{os}}_c\\
    \end{bmatrix} =
    \begin{bmatrix}
        {S_s}_s & D_s \\
        D_c & {S_s}_c \\
    \end{bmatrix}
    \begin{bmatrix}
        H_s\\
        H_c\\
    \end{bmatrix}
    \label{eq:eqmat}
\end{equation}

The equations are solved using the 2X2 matrix inverse to get \cref{eq:cal-solved}

\begin{gather}
    \begin{bmatrix}
        H_s\\
        H_c\\
    \end{bmatrix} =
    \begin{bmatrix}
        {S_s}_s & D_s \\
        D_c & {S_s}_c \\
    \end{bmatrix} ^{-1}
    \begin{bmatrix}
        V'_s - {V_{os}}_s\\
        V'_c - {V_{os}}_c\\
    \end{bmatrix}
    =
    \frac{1}{{S_s}_c \cdot {S_s}_s - D_s \cdot D_c}
    \begin{bmatrix}
        {S_s}_c & -D_s \\
        -D_c & {S_s}_s \\
    \end{bmatrix}
    \begin{bmatrix}
        V'_s - {V_{os}}_s\\
        V'_c - {V_{os}}_c\\
    \end{bmatrix}\\
    \begin{aligned}
        H_s & = \frac{{S_s}_c \left( V'_s - {V_{os}}_s \right)}{{S_s}_c \cdot {S_s}_s - D_s \cdot D_c} - \frac{D_s \left( V'_c - {V_{os}}_c \right)}{{S_s}_c \cdot {S_s}_s - D_s \cdot D_c}\\
        H_c & = - \frac{D_c \left( V'_s - {V_{os}}_s \right)}{{S_s}_c \cdot {S_s}_s - D_s \cdot D_c} + \frac{{S_s}_s \left( V'_c - {V_{os}}_c \right)}{{S_s}_c \cdot {S_s}_s - D_s \cdot D_c}
    \end{aligned}
    \label{eq:cal-solved}
\end{gather}

\Cref{eq:cal-expanded} shows \cref{eq:cal-solved} after expanding and gathering like terms.

\begin{equation}
    \begin{aligned}
        H_s & = \frac{{S_s}_c \cdot V'_s }{{S_s}_c \cdot {S_s}_s - D_s \cdot D_c} - \frac{D_s \cdot  V'_c }{{S_s}_c \cdot {S_s}_s - D_s \cdot D_c} - \frac{{S_s}_c \cdot {V_{os}}_s  -D_s \cdot {V_{os}}_c}{{S_s}_c \cdot {S_s}_s - D_s \cdot D_c}\\
        H_c & = - \frac{D_c \cdot V'_s }{{S_s}_c \cdot {S_s}_s - D_s \cdot D_c} + \frac{{S_s}_c \cdot  V'_c }{{S_s}_c \cdot {S_s}_s - D_s \cdot D_c} - \frac{{S_s}_s \cdot {V_{os}}_s  -D_c \cdot {V_{os}}_c}{{S_s}_c \cdot {S_s}_s - D_s \cdot D_c}
    \end{aligned}
    \label{eq:cal-expanded}
\end{equation}

\Cref{eq:cal-expanded} is simplified using the substitutions shown in \cref{eq:cal-sub} to get \cref{eq:magcal2}.

\begin{equation}
    \begin{aligned}
        C_1 &=   \frac{{S_s}_c}{{S_s}_c \cdot {S_s}_s - D_s \cdot D_c}\\
        C_2 &= - \frac{D_s}{{S_s}_c \cdot {S_s}_s - D_s \cdot D_c}\\
        C_3 &= - \frac{{S_s}_c \cdot {V_{os}}_s  -D_s \cdot {V_{os}}_c}{{S_s}_c \cdot {S_s}_s - D_s \cdot D_c}\\
        C_4 &= - \frac{D_c}{{S_s}_c \cdot {S_s}_s - D_s \cdot D_c}\\
        C_5 &= \frac{{S_s}_c}{{S_s}_c \cdot {S_s}_s - D_s \cdot D_c}\\
        C_6 &= - \frac{{S_s}_s \cdot {V_{os}}_s  -D_c \cdot {V_{os}}_c}{{S_s}_c \cdot {S_s}_s - D_s \cdot D_c}
    \end{aligned}
    \label{eq:cal-sub}
\end{equation}

Which gives \cref{eq:magcal} repeated here:
\Cref{eq:magcal2} is the same as \cref{eq:magcal}.

\begin{equation}
    \begin{split}
        H_s &= C_1 \cdot V_s' + C_2 \cdot V_c' + C_3\\
        H_c &= C_4 \cdot V_s' + C_5 \cdot V_c' + C_6
    \end{split}
    \label{eq:magcal2}
\end{equation}



