% vim: set filetype=tex spell :

%Introduction: give background on ARC project

\chapter{Introduction}


\section{CubeSats}

Over the past few decades electronic devices have continued to get smaller, cheaper and more powerful. This has made many things possible that were not feasible before. Recently it has become possible for universities to build and launch their own small satellites.

The \acf{CDS} is a small satellite standard that many universities, private companies and NASA are building to. The \ac{CDS} defines, among other things, the size of the CubeSats, a 10 cm cube, and how they interface with the launching mechanism which puts them into orbit. Because CubeSats are small and launched from a standard launcher they are easy to integrate into the extra space leftover on larger missions making launches relatively cheap and easy to come by.

In recent years the number of CubeSat launches has grown significantly. CubeSats are launched by universities as well as government institutions like NASA and the Air Force\todo{find references} and private companies. CubeSats were originally envisioned as an educational tool that would have the approximate capabilities of Sputnik. While CubeSats may have been conceived as primarily an educational tool\todo{find reference(s)} many CubeSats do real science\todo{find reference(s)}.

Because CubeSats are small they have limited surface area for power generation. Many CubeSats do not use deployable solar panels because of the increased complexity and weight. The result is that CubeSats have to run on a limited power budget. Many CubeSat science missions require at the very least a stable attitude or a nadir pointing attitude. With the limited power and mass budgets of CubeSats the usual attitude control methods, used on larger satellites, such as reaction wheels or thrusters use to much power mass and or volume to be feasible.

One common CubeSat attitude control method is to use the earths magnetic field. Magnetic attitude control systems are generally capable of less torque then traditional methods but are smaller and lighter making them a popular choice for many CubeSats. Because magnetic torquers produce different torque based on the angle between the torquer and the magnetic field, a completely passive system can be built. This is a popular choice for systems with low attitude control requirements as it is small, light and simple. 

In a passive system a permanent magnet on the space craft forces an axis to align with the local magnetic field. Because there is little friction in space, hysteresis material must be added to the spacecraft to dampen out oscillations. The hysteresis material dissipates energy in a changing magnetic field, reducing the angular velocity. The main drawback for a passive system is that there is no face of the satellite that is continually facing the earth. Having a face that constantly points in the nadir direction is useful for communications and for earth observation.

Active magnetic control allows for different attitude options such as nadir pointing. Many CubeSats use air core magnetic torquers built into the solar panel \ac{PCB}. The coils are made of spiral shaped traces in one or more layers of the \ac{PCB}. To increase efficiency some CubeSats use a magnetic material as a core. In both cases torque is generated only when current is flowing through the coil. 

This thesis will describe a new type of torquer that uses a hard magnetic core. The hard magnetic core acts as a permanent magnet that can be switched by pulsing a current through the core. Using this setup torque can be generated without a constant expenditure of current. This has been called the \acf{LPMT} because it can be used for active control without using as much power as traditional designs. The \ac{LPMT} is a good fit for CubeSats because of their low power usage.

\section{\acl{ARC}}

\ac{ARC} is a CubeSat built built by the \ac{UAF} \ac{SSEP} and funded by \ac{ASGP}. The main purpose of \ac{ARC} is to give students a chance to work on real space flight hardware. \ac{ARC} also has other mission objectives which are focused on testing out new hardware in the space environment The stated mission objectives are listed below.


\subsection{Mission Objectives}

%TODO: this is ripped from the proposal reword if necessary

\ac{ARC} has four main mission objectives. These are divided into \acp{EMO} and \acp{SMO}. The \acp{EMO} provide an educational opportunity for students. \ac{ARC} is primarily an educational project which is aimed at giving students an opportunity for hands on learning. The \acp{SMO} are missions where the objective is to  \cite{ArcCdr}.

\begin{description}

    \item[\acl{EMO} 1] Provide an authentic, interdisciplinary, hands-on student experiences in science and engineering through the design, development, operation of a student small satellite mission.

    \item[\acl{SMO} 1] Characterize thermal and vibration environment inside the launch vehicle from ignition to orbital insertion.

    \item[\acl{SMO} 2] Validate a novel low power \ac{ACDS}.

    \item[\acl{SMO} 3] Validate a high bandwidth communication system by obtaining images of changing snow/ice coverage in arctic regions.
\end{description}

\section{\acl{ACDS}}

The \ac{ACDS} algorithm and torquers have been designed by Dr. Donald Mentch for his PhD thesis. The remaining work for this thesis is the design, implementation and testing of the hardware and software to realize the \ac{ACDS} on the \ac{ARC}.

\ChapterRef{ch:BG} outlines the previous work on \acp{LPMT} and the alignment algorithm that will be used on the \ac{ARC}.

\ChapterRef{ch:CubeSatHardware} describes how the \ac{ACDS} is implemented on the \ac{ARC}. This hardware was constructed based on the parameters outlined in \cite{Mentch11} but the bulk of the design was done by the author of this thesis.

\ChapterRef{ch:Software} describes the software for the \ac{ACDS}. The algorithm for \ac{ACDS} was written by Donald Mentch for his Doctoral work \cite{Mentch11}. For this thesis the algorithm was taken from the Matlab simulation and converted into C code so that it could be run on a msp430. Because of the change in platform many other support functionality had to be written. Some of this functionality will be documented here.

\ChapterRef{ch:Verification} outlines the verification plan for flight hardware to make sure that it performs adequately on orbit. Some of the basic ideas of verification were outlined in \cite{Mentch11} as well as conversations with Donald Mentch. Most of the testing and verification has been performed by the author of this thesis and adjusted accordingly.

