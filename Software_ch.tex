% vim: filetype=tex spell

\chapter{Software}
\label{ch:Software}

The main responsibility of the \ac{ACDS} software is to determine which torquer to flip each time step. The \ac{ACDS} also needs to keep track of ``house keeping'' information, sensor readings, torquer states and internal control system states so that the performance of the algorithm can be tracked on the ground. 

\section{Overview}

In order to determine which torquer to flip the \ac{ACDS} needs sensor inputs. The sensors are read by the \ac{LEDL} board and forwarded to the \ac{ACDS} using the \ac{ARC}bus. The \ac{ACDS} needs to communicate with the \ac{COMM} board to receive uplinked orbital elements and respond to ground station commands.

\begin{comment}
\begin{figure}[H]
    \centering
    \begin{tikzpicture}[node distance = 3cm, auto]
        % Place nodes
        \node [block] (AB) {\acs{ARC}bus interface};
        \node [block,right of=AB] (KF) {Kalman Filter};
        \node [block,above of=KF] (alg) {\acs{ACDS} algorithm};
        \node [block,above of=AB] (CMD) {\acs{ACDS} command parse};
        \node [block,right of=alg] (TQ) {Torquer Control and state tracking};
        \node [block,right of=KF] (HC) {House Keeping};
        \node [point,below of=KF] (DN)  {};

        \path [flow] (KF) -- (alg);
        \path [flow] (AB) -- (CMD);
        \path [flow] (alg) -- (TQ);
        \path [flow] (AB) -- (KF);

        %command flow
        \path [flow] (CMD) -- (KF);
        \path [flow] (CMD) -- (alg);

        %house keeping
        \path [flow] (alg) -- (HC);
        \path [flow] (KF) -- (HC);
        \path [flow] (TQ) -- (HC);
        \path [flow] (HC) |- (DN);
        \path [flow] (DN) -| (AB);

    \end{tikzpicture}
    \caption{\acs{ACDS} software overview}
\end{figure}
\end{comment}

\begin{figure}[H]
    \centering
    \begin{tikzpicture}[node distance = 3cm, auto]
        % Place nodes
        \node [block,minimum height=8cm]            (AB) {\acs{ARC}bus interface};

        \node [block,left of=AB,node distance=6cm]  (LEDL) {\acs{LEDL}};
        \node [block,left of=AB,yshift=-3cm]        (CDH) {\acs{CDH}};
        \node [block,left of=CDH]                   (COM) {\acs{COMM}};

        \node [bigblock,right of=AB,node distance=7cm,minimum height=5cm,yshift=1.5cm] (core) {\acs{ACDS} software};
        \node [bigblock,below of=core,node distance=4.5cm,minimum height=2cm] (HC) {House Keeping};


        \path [flow] (AB.58) -- node  {Sensor Data} (core.170);
        \path [flow] (core.190) -- node {Sensor Commands} (AB.48);

        \path [flow] (AB.00) -- node {Ground Station} node [below]{Commands} (core.226);
        \path [flow] (AB.68) -- node {Orbital Elements} (core.140);

        \path [flow] (HC) -- (AB.290);

        \path [flow] (COM) -- (CDH);

        \path [flow] (CDH) -- (AB.250);

        \path [flow] (LEDL.10) -- node  {Sensor Data} (AB.170);
        \path [flow] (AB.190) -- node  {Sensor Commands} (LEDL.350);


    \end{tikzpicture}
    \caption{\acs{ACDS} software overview}
\end{figure}



\section{System Operations Overview}

\Cref{fig:sysopflow} shows the system operations flow. The code starts running after the separation switch is switched and the power system applies power to all systems. Before starting any operations the \ac{ACDS} waits for the on command from the \ac{CDH} board. After the \ac{ACDS} board receives the on command the \ac{ACDS} sends a command to the \ac{LEDL} board to tell it to start taking sensor data. Once sensor data is received the \ac{ACDS} starts to run the detumble algorithm. The Kalman filter needs to know the location of the satellite for it to operate properly so orbital elements must be uploaded before the Kalman filter can start running. Once the Kalman filter knows the location of the satellite it still needs some time to converge to a solution. During this time the \ac{ACDS} remains in detumble mode even if the rates have slowed enoughs to exit. Once the Kalman filter is locked and the rates have sufficiently slowed the \ac{ACDS} switches into mode 2. Mode two is run for a set number \todo{figure out how many and put the number here} of orbits (measured by timing). After mode 2 is complete Mode 3 starts. The \ac{ACDS} remains in mode 3 indefinably unless one of undesirable conditions is detected, see \cite{Mentch11} for details, which causes the \ac{ACDS} to attempt to kick the satellite out of the undesirable alignment and then re-stabilize by transitioning back to mode 2.

\begin{figure}[H]
    \centering
    \begin{tikzpicture}[node distance = 3cm, auto]
    % Place nodes
    \node [block] (sep) {Separation};
    \node [block,below of=sep] (cmd) {On Command from \acs{CDH}};
    \node [block,below of=cmd] (sens) {Send Start Data Collection Command to \acs{LEDL}};
    \node [point,right of=sens,node distance= 3cm]      (p3) {};
    \node [point,right of=sep,node distance= 6cm]       (p4) {};
    \node [block,below of=p4,node distance= 1cm] (detumble) {Detumble};
    \node [point,right of=detumble,node distance= 4cm] (p1) {};
    \node [decision,below of=detumble,node distance = 4cm] (dchk) {Rates and Kalman filter are ready};
    \node [block,below of=dchk,node distance=4cm] (m2) {Mode 2};
    \node [decision,below of=m2,node distance= 3.5cm] (m2chk) {Has Mode 2 time elapsed?};
    \node [point,right of=m2,node distance= 4cm] (p2) {};
    \node [block,below of=m2chk,node distance= 3.5cm] (m3) {Mode 3};
    \node [decision,below of=m3] (cchk) {Correct Alignment?};
    \node [block,right of=cchk,node distance=4cm]  (cor) {apply correction};
    \node [block,above of=cor]  (timer) {reset Mode 2 timer};

    \path[conn] (sep) -- (cmd);
    \path[conn] (cmd) -- (sens);
    \path[line] (sens) -- (p3);
    \path[line] (p3) |- (p4);
    \path[conn] (p4) -- (detumble);

    \path[conn] (detumble) -- (dchk);
    \path[conn] (dchk) --  node [near start] {yes} (m2);
    \path[line] (dchk) -|  node [near start] {no} (p1);
    \path[conn] (p1) -- (detumble);
    \path[conn] (m2) -- (m2chk);
    \path[conn] (m2chk) -- node [near start] {yes} (m3);
    \path[line] (m2chk) -| node [near start] {no} (p2);
    \path[conn] (p2) -- (m2);

    \path[conn] (m3) -- (cchk);
    \path[conn] (cchk) -- node [near start] {no} (cor);
    \path[conn] (cor) -- (timer);
    \path[conn] (timer) |- (m2);

    \path[conn] (cchk.west) -- node [near start] {yes} ([xshift=-1.5cm]cchk.west) |- (m3);

    \end{tikzpicture}
    \caption{System operations chart}
    \label{fig:sysopflow}
\end{figure}

\subsection{Deviations From the flow}

The flow case the flow from \cref{fig:sysopflow} can be disrupted. This can happen due to commands from the ground station or if the gyros detect that the sattelite is rotating too fast. The ground station can force the \ac{ACDS} into any mode and ether let it go with the normal flow or stay in a particular mode. This allows for recovery in case the system does not function as expected.

If the gyros detect that the satellite is rotating too fast then they will automatically send the \ac{ACDS} into a safe mode where the \ac{ACDS} does not run. It is possible that the \ac{ACDS} could get into a situation where the algorithm speeds up the rotation instead of slowing it down. In this situation the best thing to do is for the \ac{ACDS} to stop and wait for further intervention from the ground.

\section{Incoming Commands\textbackslash Info}

The \ac{ACDS} must respond to incoming commands that change its operation mode or uplink data. Because the \ac{ACDS} is an experimental system, it may run into unexpected problems. By uplinking commands some of the possible problems can be fixed in flight. 

In order for the Kalman filter to work it needs the orbital elements for the satellite. These must be uplinked from the ground because they change as the orbit decays. In addition gains for the control algorithm and Kalman filter may need to be changed in flight to account for unexpected conditions. 

\begin{comment}
\begin{itemize}
    \item Ground Station Commands
        \begin{itemize}
            \item Uplink Orbit Data
            \item Stop \ac{ACDS}
            \item Force Mode
        \end{itemize}
    \item Sensor Data
\end{itemize}
\end{comment}

\section{Algorithm Software}

\Cref{fig:swblock} shows the overall software block diagram for the \ac{ACDS} system. Field measurements from the magnetometer are calibrated using the current torquer state and a table of field measurements at different torquer states. The calibrated readings are used to calculate rotation rate and latitude which a long with the field readings are used to calculate the torque that should be applied to the satellite. The torque is then quantized based on \cref{fig:lpmtq}. The torquers to flip are chosen based current knowledge of torquer stated as well as the desired torque.

\begin{figure}[H]
    \centering
    \begin{tikzpicture}[node distance = 2.6cm, auto]
    % Place nodes
    \node [block] (field) {Magnetic Field Measurements};
    \node [block,right of=field] (cal) {Torquer offset correction};
    \node [point,right of=cal] (split) {};
    \node [block, right of=split,node distance=1.5cm] (alg) {Torque algorithm};
    \node [block, right of=alg] (q) {Torque Quantization};
    \node [block, above of=alg] (rates) {Kalman Filter};
    \node [block,left of=rates,node distance= 3cm] (igrf) {Magnetic field model};
    \node [block,left of=igrf] (pos) {Orbit timing};
    \node [block, below of=alg] (win) {Bias Window Determination};
    \node [point, below of=win] (push) {};
    \node [block, right of=q] (choose) {Choose Torquers to fire};
    \node [point, right of=choose] (fb) {};
    \node [block, right of=fb,node distance=1.5cm] (fire) {Fire Torquers};
    \node [block, below of=choose] (mem) {Torquer Status Tracking};
    \node [block, below of=fire] (sens) {Torquer feedback};

    %draw lines
    \path [conn] (field) -- (cal);
    \path [conn] (split) |- (rates.190);
    \path [conn] (split) |- (win);
    \path [conn] (cal) -- (alg);
    \path [conn] (rates) -- (alg);
    \path [conn] (win) -- (alg);
    \path [conn] (alg) -- (q);
    \path [conn] (q) -- (choose);
    \path [conn] (choose) -- (fire);

    \path [conn] (fb) |- (mem.10);
    \path [phconn] (fire) -- (sens);

    \path [conn] (sens.190) -- (mem.350);
    \path [conn] (mem) -- (choose);
    \path [line] (mem) |- (push);
    \path [conn] (push) -| (cal);

    \path [conn] (pos) -- (igrf);
    \path [conn] (igrf.10) -- (rates.170);


    \end{tikzpicture}
    \caption{Overall Software Block Diagram}
    \label{fig:swblock}
\end{figure}

\subsection{Mode 1}

\begin{comment}
\Cref{fig:mode1} shows the Mode 1 torque algorithm block diagram. In Mode 1 the required torque is simply calculated using rotation rates and field measurements with \cref{eqn:crossl}. This is also sometimes referred to as the detumble phase because the tumbling motion of the satellite is slowed down to a rate that makes it easier to get into the proper alignment.

\begin{figure}[H]
    \centering
    \begin{tikzpicture}[node distance = 3cm, auto]
    % Place nodes
    \node [input] (field) {Magnetic Field};
    \node [block, right of=field] (alg) { $k {{\vect{\omega}_{err} \cross \vect{B}} \over{\vect{B} \cdot \vect{B}}}$ };
    \node [input, right of=alg] (rates) {Rotation Rates};
    \node [point, below of=alg] (out) {};

    %draw lines
    \path [conn] (field) -- (alg);
    \path [conn] (rates) -- (alg);
    \path [conn] (alg) -- (out);


    \end{tikzpicture}
    \caption{Mode 1 Torque Algorithm Block Diagram}
    \label{fig:mode1}
\end{figure}
\end{comment}

\Cref{fig:mode1} shows the Mode 1 torque algorithm block diagram. In Mode 1 the required torque is calculated using the derivative of the magnetic field. This is a method that is widely used on other CubeSats \todo{Find some B-dot reference(s)} that have magnetic \ac{ACDS}. In this mode the magnetic dipole moment is simply set to a value that is proportional to the derivative of the magnetic field. Because the knowledge of the rotation rates is not needed for Mode 1, detumble can start before the Kalman filter has locked. The gain for the B-dot controller needs to be negative in order to stop the motion of the magnetic field. This could be done by adding a negative sign to the torque expression but, in code, it is simpler to have a negative gain.

\begin{figure}[H]
    \centering
    \begin{tikzpicture}[node distance = 3cm, auto]
    % Place nodes
    \node [input] (field) {Magnetic Field};
    \node [block, right of=field] (alg) { $C {\dot{\vect{B}}}$ };
    \node [point, right of=alg] (out) {};

    %draw lines
    \path [conn] (field) -- (alg);
    \path [conn] (alg) -- (out);


    \end{tikzpicture}
    \caption{Mode 1 Torque Algorithm Block Diagram}
    \label{fig:mode1}
\end{figure}

\subsubsection{MSP430 Implementation}

\todo[inline]{Implement this and write about it}

\subsection{Mode 2}

\Cref{fig:mode2} shows the Mode 2 torque algorithm block diagram. Torque is calculated the same way as in Mode 1 but this time a bias is added depending on which region of the orbit the satellite is in. The bias tends to cause the satellite to rotate. This causes the algorithm to cancel out the bias. This is prevented by preventing the resulting dipole moment from being in the opposite direction as the bias. Outside of the bias regions the torquers are set to produce no torque.

\begin{figure}[H]
    \centering
    \begin{tikzpicture}[node distance = 3cm, auto]
    % Place nodes
    \node [input] (field) {Magnetic Field};
    \node [block, right of=field] (alg) { $k {{\vect{\omega}_{err} \cross \vect{B}} \over{\vect{B} \cdot \vect{B}}}$ };
    \node [input, right of=alg] (rates) {Rotation Rates};
    \node [oppr,below of=alg,node distance=2cm] (sum) {+};
    \node [block,left of=sum,node distance=3cm] (bias) {Mode 2 bias table};
    \node [input,left of=bias] (win) {Bias Window};
    \node [block,below of=sum,node distance=2cm] (fix) {Bias Fix};
    \node [block,below of=fix,node distance=2.5cm] (coast) {Coast?};
    \node [point, below of=coast] (out) {};

    %draw lines
    \path [conn] (field) -- (alg);
    \path [conn] (rates) -- (alg);
    \path [conn] (alg) -- (sum);
    \path [conn] (win) -- (bias);
    \path [conn] (bias) -- (sum);
    \path [conn] (sum) -- (fix);
    \path [conn] (fix) -- (coast);
    \path [conn] (win) |- (coast);
    \path [conn] (coast) -- (out);
    \path [conn] (bias) |- (fix);

    \end{tikzpicture}
    \caption{Mode 2 Torque Algorithm Block Diagram}
    \label{fig:mode2}
\end{figure}

\subsection{Mode 3}

\Cref{fig:mode3} shows the Mode 3 torque algorithm block diagram. This is similar to Mode 2 except only the north pole bias window is used and outside the bias window torque is generated to slow rotation rates.

\begin{figure}[H]
    \centering
    \begin{tikzpicture}[node distance = 3cm, auto]
    % Place nodes
    \node [input] (field) {Magnetic Field};
    \node [block, right of=field] (alg) { $k {{\vect{\omega}_{err} \cross \vect{B}} \over{\vect{B} \cdot \vect{B}}}$ };
    \node [input, right of=alg] (rates) {Rotation Rates};
    \node [oppr,below of=alg,node distance=2cm] (sum) {+};
    \node [block,left of=sum,node distance=3cm] (bias) {Mode 3 bias table};
    \node [input,left of=bias] (win) {Bias Window};
    \node [block,below of=sum,node distance=2cm] (fix) {Bias Fix};
    \node [point, below of=fix] (out) {};

    %draw lines
    \path [conn] (field) -- (alg);
    \path [conn] (rates) -- (alg);
    \path [conn] (alg) -- (sum);
    \path [conn] (win) -- (bias);
    \path [conn] (bias) -- (sum);
    \path [conn] (sum) -- (fix);
    \path [conn] (fix) -- (out);
    \path [conn] (bias) |- (fix);


    \end{tikzpicture}
    \caption{Mode 3 Torque Algorithm Block Diagram}
    \label{fig:mode3}
\end{figure}

\subsection{Mode Switching}

Mode switching on \ac{ARC} will be time based. The original simulation used the rotation rates to switch from Mode 1 mode to Mode 2 but it was found that when the satellite did not have full knowledge of rotation rates, as will be the case if they are determined from magnetic field, then the mode was switched too soon. The switch from Mode 2 to Mode 3 is timed to be about 10 orbits after the switch into Mode 2. Once in Mode 3 no more automated mode switching is done. Modes can also be switched at all times via ground station command.

\begin{comment}
\subsection{Bias Window Determination}

\Cref{fig:biaswin} shows how the bias window determination will work. To determine where \ac{ARC} is within the orbit, peaks are found in the magnitude of the magnetic field. The field magnitude is used because it is independent of attitude. The magnitude of the field is greatest at the south pole but the magnetic north pole is located closer to the geographic north pole so it's peaks are more consistent. South pole peaks are filtered out first by eliminating peaks that are ether too high or too low to be a north pole peak. Peaks are further filtered by tracking the timing of past peaks and rejecting peaks that happen too soon. If a second peak is detected very soon after a north pole peak then it is assumed to be a double peak and the average peak time is taken. After the north poles are found the orbital period can be determined and orbital progress can be tracked. The current bias window is determined using the current orbital progress and a lookup table. In case the automatic orbit tracking does not work properly it is possible to upload timing parameters for the orbit tracking to use instead.

\begin{figure}[H]
    \centering
    \begin{tikzpicture}[node distance = 3cm, auto]
    % Place nodes
    \node [input] (field) {Field Measurements};
    \node [block,below of=field,node distance=2.1cm] (mag) {Vector Magnitude};
    \node [block,right of=mag] (peak) {Peak Detect};
    \node [block,right of=peak] (level) {Level Check};
    \node [block,right of=level] (pole) {North Pole Detect};
    \node [block,right of=pole] (gen) {Orbit tracking};
    \node [block,below of=gen,node distance=2.5cm] (time) {Timing Source};
    \node [block,above of=gen] (upload) {Timing Parameter Upload};
    \node [block,right of=gen] (lookup) {Bias Window Lookup};
    \node [point,right of=lookup,node distance=2cm] (out) {};

    %draw lines
    \path [conn] (field) -- (mag);
    \path [conn] (mag) -- (peak);
    \path [conn] (peak) -- (level);
    \path [conn] (level) -- (pole);
    \path [conn] (pole) -- (gen);
    \path [conn] (gen) -- (lookup);
    \path [conn] (lookup) -- (out);
    \path [conn] (time) -- (gen);
    \path [conn,dashed] (upload) -- (gen);

    \end{tikzpicture}
    \caption{Bias Window Determination Block Diagram}
    \label{fig:biaswin}
\end{figure}

\end{comment}

\begin{comment}
\subsection{Rotation Rate Calculations}

The rotation rate is calculated using the magnetic field measurements. \Cref{eqn:magrate} shows how to calculate the rotation rate using magnetic field measurements. In \cref{eqn:magrate} $\dot{\vec{B}}$ is calculated using there magnetic field measurements and the central difference formula as suggested in \cite{Mentch11}.

\begin{equation}
    %TODO: add dot over B
    \vec{\omega}={\dot{\vec{B}} \cross \vec{B} \over{\vec{B} \cdot \vec{B}}}
    \label{eqn:magrate}
\end{equation}
\end{comment}

\subsection{Kalman Filter}

\begin{figure}[H]
    \centering
    \begin{tikzpicture}[node distance = 3cm, auto]
    % Place nodes
        \node [block] (pstate) {Project State};
        \node [block,below of=pstate] (pcov) {Project Error Covariance Ahead};
        \node [block,right of=pcov] (gain) {Compute Kalman Gain};
        \node [block,right of=gain] (ustate) {Update state Estimate};
        \node [input,right of=ustate,text width = 3cm,node distance=3.5cm] (meas) {Measurement Input};
        \node [block,above of=ustate] (ucov) {Update Error Covariance};


    %draw lines

    \path [conn] (pstate) -- (pcov);
    \path [conn] (pcov) -- (gain);
    \path [conn] (gain) -- (ustate);
    \path [conn] (meas) -- (ustate);
    \path [conn] (ustate) -- (ucov);
    \path [conn] (ucov) -- (pstate);


    \end{tikzpicture}
    \caption{Extended Kalman Filter}
    \label{fig:eKalman}
\end{figure}

\section{Auxiliary Software Functions}

In addition to the algorithm software the \ac{ACDS} also requires some auxiliary software to function. In order to generate the proper dipole moment the status of the torquers must be tracked in software so that the right torquer can be flipped. The torquer status also needs to be tracked so that the torquer offsets can be subtracted from the magnetometer measurements. 

\subsection{Torquer Flipping Logic}

The torquer flipping logic keeps track of the torquer status and sets the torquers to the dipole moment that is closest to the dipole moment requested by the control algorithm. The torquer flipping logic also attempts to distribute the flips somewhat evenly across the torquers in any given axis. This way if there are any degradation effects, in the torquer field or the hardware, they will be minimized. 

\subsubsection{Status Tracking}

The status of each torquer is tracked by the flipping logic. This includes flags to indicate if each torquer has been flipped and has had errors being flipped. The last four torquers that were flipped are also tracked. This is used to chose which torquer to flip.

When it is time to flip a torquer the software looks at the current torquer status to figure out which direction a torquer needs to be flipped in. Next the software looks for the torquer that can be flipped in that direction that was flipped the least recently and flips it. This way the torquer flips are distributed across the torquers in a given axis. If torquers are marked as uninitialized then they are flipped first before other torquers so that all torquers are in a known state as quickly as possible.

\subsubsection{torquer feedback}

Before and after each torquer flip the torquer feedback comparators are read. After the torquer flip is complete the feedback values are examined to determine if the torquer flipped. If the capacitor voltage was below the upper threshold before the flip then an error is flagged in the torquer status indicating that there is a problem with the capacitor. If the capacitor voltage after a flip is not below the lower threshold then there is likely a bad connection to the torquer windings. Torquers that failed to flip are flagged and not flipped in the future. It is possible that the comparator could malfunction and indicate that it is both above the upper threshold and below the lower threshold in this case a flag is set to indicate the error and the torquer is assumed to have flipped as normal.

\subsection{Torquer Compensation}

Because the geometry of the system should not change in flight, the field offset seen by each magnetometer depends on the combination of torquer states. By taking measurements in the Helmholtz cage these offsets can be calculated and eliminated during flight in order to measure Earth's magnetic field.

\subsubsection{Compensation Data Set}

There are a total of 12 \acp{LPMT} on the \ac{ACDS}. Each \ac{LPMT} has two states which results in a total of 4096 possible states. If the offset for each state uses four bytes this results in a data set that is 16kB. The field produced at each state however, is the sum of the field contributions of all of the torquers. The dataset can be reduced by doing three separate calibration, one for the set of torquers in each axis. The calibrations are then combined to get a calibration set where one offset is chosen for each axis and added together to get the full offset this reduces the number of states to $3*16 + 1 = 49$ which is only about 200 bytes of data, much reduced from using all possible states.

\subsubsection{Compensation Routine}

To calculate the compensation values for the torquers the calibration procedure described in \cref{sec:magcal} is run for each combination of torquer states. The results of the compensation is $C_1$, $C_2$, $C_4$ and $ C_5$ which are the same calibration constants computed with the magnetometer calibration routine. In addition there is one common pair of offsets and then 3 sets of 16 offsets for each set of torquers.


\subsection{Data Logging}

During \ac{ACDS} operations data is collected during flight. This data is stored in nonvolatile memory and can be recalled and transmitted to the ground in order to evaluate the \ac{ACDS} system performance. The data that is recorded is shown in \cref{tab:logdat}

\begin{comment}
\begin{itemize}
    \item magnetometer and gyro readings
    \item torquer status
    \item mode
    \item algorithm intermediate results
    \item torquers flipped
    \item torquer feedback
    \item Kalman filter status
    \item Kalman filter internal variables
    \item Kalman filter state
    \item \todo[inline]{More?}
\end{itemize}
\end{comment}

\begin{table}[H]
    \centering
    \caption{\ac{ACDS} operations data format}
    \label{tab:logdat}
    \begin{tabular}{|l|c|c|}
        \hline
        Data&size (bits)&Format\\
        \hline
        mode&2&unsigned integer\\
        \hline
        Time Stamp&32&unsigned integer\\
        \hline
        torquer status&48&flags\\
        \hline
        magnetometer readings&48&unsigned integer (one per axis)\\
        \hline
        gyro readings&36&unsigned integers (one per axis)\\
        \hline
        algorithm intermediate results&TBD&\\
        \hline
        torquers flipped&48&unsigned integer (one per axis)\\
        \hline
        torquer feedback&12&flags\\
        \hline
        Kalman filter status&TBD&\\
        \hline
        Kalman filter internal variables&TBD&\\
        \hline
        \multicolumn{1}{|r|}{\bfseries Total :}&TBD&\\
        \hline
    \end{tabular}
    \todo[inline]{Resolve DBD's}
\end{table}

\subsection{On Board data processing}

Because the downlink data speed is limited, it is necessary to reduce the data that needs to be downlinked. One way to do this is to do some level of on board processing to reduce the data before it is downlinked. For the \ac{ACDS} system this will most likely take the form of returning only the desired data from the recorded data set or returning min/max values from within a data set. 

\subsection{Beacon Data}

The beacon data is transmitted at a fixed rate and provides information about the present state of the satellite. The beacon data could be the only data that is available to diagnose the system. The beacon data must contain enough data to be useful but not so much data that the beacon packets are too big. The beacon data should include raw sensor information as well as information on torquer status. The beacon data from the \ac{ACDS} is shown in \cref{tab:beacondat}

\begin{table}[H]
    \centering
    \caption{Beacon Data format}
    \label{tab:beacondat}
    \begin{tabular}{|l|c|c|}
        \hline
        Data&size (bytes)&Format\\
        \hline
        current magnetometer readings&6&signed integers (one per axis)\\
        \hline
        Mode&1&unsigned integer\\
        \hline
        current torquer status&3&flags\\
        \hline
        number of torquer flips so far & 6 & unsigned integers (one per axis)\\
        \hline
        Kalman filter attitude&8&integer quaternion\\
        \hline
        Kalman filter rates&6&integer vector\\
        \hline
        \multicolumn{1}{|r|}{\bfseries Total :}&30&\\
        \hline
    \end{tabular}
\end{table}

